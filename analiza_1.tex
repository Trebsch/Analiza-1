%%%%%%%%%%%%%%%%%%%%%%%%%%%%%%%    PRIPRAVE STY STVARI 1    %%%%%%%%%%%%%%%%%%%%%%%%%%%%%%%%%%%%%%%
\documentclass[12pt, a4paper, unicode]{article}
\usepackage{Priprave}

\newcommand{\naslov}{Analiza 1}
%%%%%%%%%%%%%%%%%%%%%%%%%%%%%%%%%%%%%%%%%%%%%%%%%%%%%%%%%%%%%%%%%%%%%%%%%%%%%%%%%%%%%%%%%%%%%%%%%%%%

%%%%%%%%%%%%%%%%%%%%%%%%%%%%%%%%%%% IME, MAIL IN NASLOV %%%%%%%%%%%%%%%%%%%%%%%%%%%%%%%%%%%%%%
\newcommand{\ime}{Hugo Trebše}
\newcommand{\mail}{hugo.trebse@gmail.com}
%\newcommand{\podnaslov}{Podnaslov}
%Če želiš podnaslov rabiš it v .sty pa uncommentat razdelek v sectionu %NASLOV IN PODNASLOV%
%%%%%%%%%%%%%%%%%%%%%%%%%%%%%%%%%%%%%%%%%%%%%%%%%%%%%%%%%%%%%%%%%%%%%%%%%%%%%%%%%%%%%%%%%%%%%

\addbibresource{main.bib}

%%%%%%%%%%%%%%%%%%%%%%%%%%%%%%  NOVE MATEMATIČNE OZNAKE %%%%%%%%%%%%%%%%%%%%%%%%%%%%%%%%%
\newcommand{\vp}{\nu_p}
\renewcommand{\P}{\mathbb{P}}
\DeclareMathOperator{\lcm}{lcm}
\newcommand{\ed}{\forall \varepsilon >0$.$\: \exists \delta>0$.$\:}
%%%%%%%%%%%%%%%%%%%%%%%%%%%%%%%%%%%%%%%%%%%%%%%%%%%%%%%%%%%%%%%%%%%%%%%%%%%%%%%%%%%%%%%%%%%%%%%%%%

%%%%%%%%%%%%%%%%%%%%%%%%%%%%%%    ZA CITATE    %%%%%%%%%%%%%%%%%%%%%%%%%%%%%%%%%%%%%%%%%
\usepackage{epigraph}
\renewcommand{\epigraphflush}{center}
%%%%%%%%%%%%%%%%%%%%%%%%%%%%%%%%%%%%%%%%%%%%%%%%%%%%%%%%%%%%%%%%%%%%%%%%%%%%%%%%%%%%%%%%%%%%%%%%%


\begin{document}

%%%%%%%%%%%%%%%%%%%%%%%%%%%%%%   PRIPRAVE STY STVARI 2: %%%%%%%%%%%%%%%%%%%%%%%%%%%%%%%%%%%%%%%%%
\renewcommand{\headheight}{20pt}

\pagestyle{Priprave}
%%%%%%%%%%%%%%%%%%%%%%%%%%%%%%%%%%%%%%%%%%%%%%%%%%%%%%%%%%%%%%%%%%%%%%%%%%%%%%%%%%%%%%%%%%%%%%%%%

\maketitle

\newpage

\tableofcontents

\newpage

\section{Dve topološki lemi}
Preden začnemo, navedimo dve uporabni ">topološki"< lemi.

\begin{lema}[Vloženi intervali]
Naj bo $\{I_n\}_{n \in \N}$ zaporedje intervalov, za katerega velja $$I_{n+1} \subseteq I_n \quad  \text{ter} \quad \lim_{n \to \infty} \text{len}(I_n) = 0,$$ potem obstaja edinstvena točka $\lambda \in \R$, da je $$\forall n \in \N : \lambda \in I_n .$$
\end{lema}

\begin{lema}[Kompaktnost zaprtega intervala]
Vsako pokritje zaprtega intervala ima končno podpokritje.
\end{lema}
\begin{oris}
Supremum množice točk, ki jih lahko dosežemo z končno unijo okolic.
\end{oris}

\newpage
\section{Limite}

\subsection{Zaporedja}

\begin{izrek}[Supremum]
Naj bo $A$ poljubna številska množica. Potem je $s$ \textit{supremum} $A$, če je:
\begin{itemize}
    \item $s$ je zgornja meja $A$
    \item $\forall \epsilon > 0$ velja, da $s - \epsilon$ ni zgornja meja $A$
\end{itemize}
\end{izrek}

\begin{definicija}[Limes superior]
$s$ je $\limsup_{n\to \infty} a_n$, če:
\begin{itemize}
    \item $a_n < s$ velja za neskončno $n \in \N$
    \item $a_n > s$ za kvečjemu končno $n \in \N$
\end{itemize}
\end{definicija}

\begin{nasvet}
Standardni načini dokazovanja konvergence zaporedja so:
\begin{itemize}
    \item \textbf{Definicija}: $\exists A \in \R. \forall \epsilon > 0. \exists \dots$
    \item \textbf{Cauchyjeva last}: $\forall \epsilon > 0. \exists N \in \N$. $\forall m,n > N: \abs{a_m - a_n} < \epsilon$
    \item \textbf{Monotone convergence}: Vsako omejeno monotono zaporedje je konvergentno.
    \item \textbf{Izrek o Sendviču}: $a_n \leq b_n \leq c_n$. $$\lim_{n \to \infty} a_n = \lim_{n \to \infty} c_n = L \implies \lim_{n \to \infty} b_n = L.$$
\end{itemize}
\end{nasvet}
Konvergentna zaporedja lahko členoma seštevamo ter množimo ter ohranjamo limite, delimo pa lahko le v primeru neničelne limite.

\subsubsection{">Orodja"<}
\begin{izrek}[AG za zaporedja]
Če je $\{x_n\}_{n \in \N}$ zaporedje pozitivnih\footnotemark števil, ki konvergira k $X$, potem naslednji zaporedji konvergirata k $X$:
$$ \left\{ \sqrt[n]{x_1 x_2 \dots x_n} \right\}_{n \in \N} \quad \text{ter} \quad \left\{ \frac{x_1 + x_2 + \dots + x_n}{n} \right\}_{n \in \N}$$
\end{izrek}
\footnotetext{Pogoj pozitivnosti ni potreben za konvergenco zaporedja aritmetičnih sredin.}
\begin{oris}
By telescoping, za vsak $\varepsilon$ obstaja $k$, da je $\abs{\frac{1}{n}\sum_{i=k}^n x_i - X} < \varepsilon$ za vse $n >k$, hkrati za nek $n$ $\abs{\frac{1}{n}\sum_{i=1}^{k-1} x_i} < \varepsilon$, zaključimo z trikotniško neenakostjo.
\end{oris}


\begin{izrek}
Naj bo $\{x_n\}_{n \in \N}$ zaporedje pozitivnih števil. Potem za $L \in [0, \infty]$:
$$\lim_{n \to \infty} \frac{x_{n+1}}{x_n} = L \implies \lim_{n \to \infty} \sqrt[n]{x_n} = L$$
\end{izrek}

Obratna implikacija zgornjega izreka ne velja, kar dokaže zaporedje $x_n = n^{-n}$.

\begin{izrek}
Če za zaporedji realnih števil velja $\lim_{n \to \infty}a_n = 1$ in $\lim_{n \to \infty}b_n = \infty$, potem velja: $$\lim_{n \to \infty}a_n^{b_n} = \lim_{n \to \infty}e^{(a_n - 1)b_n}$$
\end{izrek}

\begin{izrek}[Stirlingova formula]
$$\lim_{n \to \infty} \frac{\sqrt{2 \pi n} \left( \frac{n}{e}\right)^n}{n!} = 1$$    
\end{izrek}

\begin{izrek}[Stolz–Cesàro]
Naj bo $\lim_{n \to \infty} a_n = \lim_{n \to \infty} b_n = 0$, ter $\{b_n\}$ padajoče (od nekega indeksa naprej).\footnotemark \newline Če obstaja prva izmed naslednjih limit sledi:
$$\lim_{n \to \infty} \frac{a_{n+1} - a_n}{b_{n+1} - b_n} = \lim_{n \to \infty} \frac{a_n}{b_n}$$
\end{izrek}
\footnotetext{Analogna trditev velja tudi v primeru $\lim_{n \to \infty} b_n = \infty$ ter $\{b_n\}$ naraščajoče od nekega indeksa dalje. \indent \indent  V tem primeru ni potrebno niti, da je $\{a_n\}$ konvergentna.}


\newpage
\subsection{Vrste}

\begin{nasvet}
Standardni načini dokazovanja konvergence številske vrste so:
\begin{itemize}
    \item \textbf{Po definiciji}: Številska vrsta konvergira $\iff$ konvergira zaporedje delnih vsot.
    \item \textbf{Cauchy}: Številska vrsta konvergira $\iff$ zaporedje delnih vsot je Cauchy.
%$\forall \epsilon > 0 \exists N \in \N : \forall m > n > N : \abs{\sum_{i = n}^{m} a_i} < \epsilon$
    \item \textbf{Primerjalni kriterij}: Če $0 \leq a_n \leq b_n$, potem:
        \begin{itemize}
            \item[$\dagger$] $\sum_{i = 1}^{\infty}b_n$ konvergira $\implies \sum_{i = 1}^{\infty}a_n$ konvergira.
            \item[$\dagger$] $\sum_{i = 1}^{\infty}a_n$ divergira $\implies \sum_{i = 1}^{\infty}a_n$ divergira.
        \end{itemize}
    \item \textbf{Potreben pogoj za konvergenco vrst}: \newline Če $\lim_{n \to \infty} a_n \neq 0$, potem vrsta $\sum_{n = 1}^{\infty} a_n$ divergira.
\end{itemize}
\end{nasvet}

Za pogoste vrste lahko parametriziramo konvergenco:
\begin{izrek}
Naslednji vrsti konvergirata za $q > 1$ ter divergirata za $q \leq 1$:
$$\sum_{n=1}^{\infty} \frac{1}{n^p} \quad \text{ter} \quad \sum_{n=1}^{\infty} \frac{1}{q^n}.$$
\end{izrek}

\begin{nasvet}
Malo manj standardni postopki za evalvacijo številskih vrst:
\begin{itemize}
    \item Interpretiraj vrsto kot Riemmanovo vsoto.
\end{itemize}
\end{nasvet}

\subsubsection{">Orodja"<}
Pri dokazovanju konvergence številskih vrst se lahko poslužimo tudi naslednjih ">orodij"<:


\begin{izrek}[Cauchyjev kriterij za konvergenco vrste]
%, \quad d_i = \frac{a_{i+1}}{a_i}, \quad r_i = i \cdot \left( \frac{a_i}{a_i+1} - 1 \right)$
Naj bo $\sum_{n = 1}^{\infty} a_n$ vrsta z pozitivnimi členi. Sestavimo zaporedje: $c_i = \sqrt[n]{a_i}$\footnotemark .

\begin{itemize}
    \item Če obstaja $q < 1$, da velja $c_i \leq q < 1$, potem  $\sum_{n = 1}^{\infty} a_n$ konvergira
    \item Če velja $d_i \geq 1$, potem $\sum_{n = 1}^{\infty} a_n$ divergira.
\end{itemize}
Če $\{c_i\}_{i \in \N}$ konvergira k $L < 1$ (oz. $L > 1$), potem vrsta konvergira (divergira).
\end{izrek}
\footnotetext{Analogna trditev velja za zaporedje $d_i = \frac{a_{i+1}}{a_i}$}
\begin{dokaz}
Primerjamo z geometrijsko vrsto.
\end{dokaz}


\begin{izrek}[Raabov kriterij za konvergenco vrste]
Naj bo $\sum_{n = 1}^{\infty} a_n$ vrsta z pozitivnimi členi. Sestavimo zaporedje: $r_i = i \cdot \left( \frac{a_i}{a_{i+1}} - 1 \right)$.
\begin{itemize}
    \item Če obstaja $q > 1$, da je $r_i \geq q > 1$, potem $\sum_{n = 1}^{\infty} a_n$ konvergira
    \item Če velja $r_i \leq 1$, potem $\sum_{n = 1}^{\infty} a_n$ divergira.
\end{itemize}
Če obstaja $r = \lim_{i \to \infty} r_i$, potem za $r > 1$ (oz. $r < 1$) vrsta konvergira (divergira).
\end{izrek}
\begin{dokaz}
Izrek dokažemo za prvi dve točki, v limitnem primeru je analogen. 

Denimo $r_i \geq q > 1$. 
Sledi: $\forall i \in \N: (a_i - a_{i+1}) \geq q \cdot a_{i+1}$. Seštejmo $n$ takih neenakosti:
\begin{align*}
\sum_{i=1}^{n} i(a_i - a_{i+1}) &\geq q\sum_{i=2}^{n+1}a_i\\
\left(\sum_{i=1}^{n} a_i \right) - n \cdot a_{n+1} &\geq q \sum_{i=2}^{n+1} a_i\\
\left(\sum_{i=1}^{n+1} a_i \right) + a_1 - (n+1) \cdot a_{n+1} &\geq q \sum_{i=1}^{n+1} a_i \\
1 + \frac{a_1}{\sum_{i=1}^{n+1}a_i} - \frac{(n+1)a_{n+1}}{\sum_{i=1}^{n+1}a_i} &\geq q\\
1 + \frac{a_1}{\sum_{i=1}^{n+1}a_i} &\geq q\\
\end{align*}
Če je $\sum_{i=1}^{\infty}a_i = \infty$ se leva stran neenakosti približa $1$ poljubno blizu, kar je v protislovju z $q > 1$. Drugo točka sledi, saj je tedaj izraz na levi v 4. vrstici neenakosti manjši od $1$, kar implicira $\frac{a_1}{n+1} \leq a_{n+1}$, kar dokaže divergentnost po primerjalnem kriteriju.
\end{dokaz}

\begin{izrek}[Cauchyjev kondenzacijski test]
Naj bo $\{a_n\}_{n \in \N}$ strogo padajoče zaporedje nenegativnih realnih števil. Potem:
$$\sum_{n=1}^{\infty} a_n \text{ konvergira} \iff \sum_{n=1}^{\infty} 2^n a_{2^n} \text{ konvergira}.$$
\end{izrek}

\newpage
\subsection{Vrste z negativnimi členi}

\begin{definicija}
Vrsta je \textit{absolutno konvergentna}, če konvergira vrsta z absolutno vrednostjo členov ter \textit{pogojno konvergentna}, če konvergira, a ne absolutno.
\end{definicija}

\begin{izrek}[Leibnizev kriterij]
Naj bo $\{a_n\}_{n \in \N}$ padajoče zaporedje z limito 0. Potem konvergira vrsta $$\sum_{i = 1}^{\infty} (-1)^i a_i.$$
\end{izrek}



\subsubsection{Preureditve vrst}
Vsaka preureditev absolutno konvergentne vrste je absolutno konvergentna k isti limiti kot začetna vrsta.

Za pogojno konvergentne vrste velja sledeče:
\begin{izrek}[Riemmanov preureditveni izrek]
Če je $\sum_{n = 1}^{\infty} a_n$ pogojno konvergentna, potem za vsak $A \in \overline{\R} \footnotemark$ obstaja bijekcija $\tau: \N \to \N$, da je: $$\sum_{n = 1}^{\infty} a_{\tau(n)} = A$$
\end{izrek}
\footnotetext{$\overline{\R} = \R \cup \{-\infty, \infty\}$}
\begin{oris}
Definiramo $p_k$ ter $n_k$ zaporedoma kot vsoto pozitivnih ter nasprotnih vrednosti negativnih členov do $k$-tega indeksa. 

Velja $s_k = p_k - n_k$ ter da $p_k + n_k$ divergira, sledi, da divergirata $$\sum_{i = 1}^{\infty}p_i \quad \text{ter} \quad \sum_{i=1}^{\infty}n_i .$$

Sedaj konstruiramo zaporedje $\{m_1, m_2, \dots\}$, kjer je $m_1$ najmanjše naravno število, da je $\sum_{i = 1}^{m_1}p_i > A$, $m_2$ najmanjše, da je $\sum_{i = 1}^{m_1}p_i - \sum_{i = 1}^{m_2}n_i < A$ ter analogno naprej. 

Bijekcija se ponuja sama, prav tako je limita očitna.
\end{oris}


\newpage
\section{Zveznost}
\begin{nasvet}[Karakterizacije zveznosti]
\begin{itemize}
    \item \textbf{Definicija:} $\forall \varepsilon >0$.$\: \exists \delta>0 \dots$ 
    \item \textbf{Zaporedja}: $f$ zvezna v $a \iff \forall \: \{a_n\} \xrightarrow{} A : \{f(a_n)\} \xrightarrow{} f(A)$
    \item \textbf{Cauchyjev pogoj}: $f$ zvezna v $a$ ter ima tam limito $\iff f$ je Cauchy pri $a$: \newline $\forall \varepsilon >0$.$\: \exists \delta>0$.$\: \forall x,x' \in D_f :$
    $$\abs{x-a} < \varepsilon \land \abs{x' - a} < \varepsilon \implies \abs{f(x) - f(x')} < \delta$$
\end{itemize}
\end{nasvet}

Pogosto je najlažje pokazati \textit{zveznost} z uporabo $\varepsilon - \delta$ delfinicije, \textit{nezveznost} pa z konstruiranjem zaporedja, katerega slike ne konvergirajo k sliki limite.

\begin{nasvet}[Uporaba zveznosti]
\begin{itemize}
    \item $f$ zvezna na kompaktu $\implies$ $f$ enakomerno zvezna, kar pomeni: \newline $\ed x,x' \in D_f$.$$\abs{x' - x} < \delta \implies \abs{f(x') - f(x)} < \varepsilon$$
    \item Zvezna funkcija na kompaktu je omejena ter doseže vse vrednosti slike.
\end{itemize}
\end{nasvet}
\begin{oris}
Dokažemo le drugo točko, prva je hitra posledica kompaktnosti zaprtega intervala. 
Omejenost sledi, ker obstaja zaporedje s slikami proti neskončno, to zaporedje je omejenno, kar pomeni da ima konvergentno podzaporedje, doseganje vrednosti med min. in max. sledi by considering $s = \sup\{x \in D_f | f(x) \leq c\}$ ter $f(s)$. Za minimum ter maksimum uvedemo pomožno funkcijo $f(x) = \frac{1}{f(x) - M}$, kjer je $M = \sup\{f(y) | y \in D_f\}$, ki je zvezna na $D_f$, posledično omejena, kar vodi v protislovje.
\end{oris}
\begin{nasvet}[Dokazovanje zveznosti]
\begin{itemize}
    \item Zveznost je lokalna lastnost $\implies$ lahko se omejimo na poljubno majhno okolico.
    \item $f$ ima \textbf{omejen odvod} v okolici $a$ $\implies$ $f$ je enakomerno zvezna v $a$.
    \item $\forall \varepsilon > 0$ je $f$ zvezna na $(a + \varepsilon, b) \implies f$ zvezna na $(a, b)$ 
\end{itemize}
\end{nasvet}

\newpage
\subsection{Limite funkcij}
\begin{izrek}[L'Hospitalovo pravilo]
Naj sta $f,g$ odvedljivi na $(a,b)$\footnotemark[2] ter $\forall x \in (a,b): g(x) \neq 0 \land g'(x) \neq 0$. Naj bo $\lim_{x \downarrow a} f(x) = \lim_{x \downarrow a} g(x) = 0$. Če obstaja prva izmed naslednjih limit sledi:\[\lim_{x \downarrow a} \frac{f'(x)}{g'(x)} = \lim_{x \downarrow a} \frac{f(x)}{g(x)}\]
\end{izrek}
Analogen izrek velja ko je $\lim_{x \downarrow a} g(x) = \pm \infty$, tedaj ne zahtevamo obstoja limite $f$ v $a$.

\begin{izrek}
Naj bo $\lim_{x \to a} f(x) = 0$ ter $\lim_{x \to a} g(x) = \infty$. Potem: \[\lim_{x \to a} \left( 1 + f(x)\right)^{g(x)} = e^{\lim_{x \to a} f(x)g(x)}\]
\end{izrek}
\begin{oris}
Sendvič z limito, ki definira $e$.
\end{oris}
\footnotetext[2]{Pozor, odvoda funkcij morata biti definirana na neki (enostranski) okolici $a$, definiranost odvodov samo v $a$ ni zadosti.}

\newpage
\section{Odvod}
\begin{definicija}
Odvod je limita diferenčnega kvocienta, difernciabilnost (ki je ekvivalentna odvedljivosti) uvedemo zgolj za dokaz formule odvoda kompozituma.
\end{definicija}

\begin{naloga}
\[\text{arctan}'(x) = \frac{1}{1+x^2} \quad \text{ter} \quad \text{arcsin}'(x) = \frac{1}{\sqrt{1-x^2}}\]
\end{naloga}

\begin{lema}
\[\frac{d}{dx} f^{-1}(x) = \frac{1}{f'(f^{-1}(x))} \quad \text{ter} \quad \left(\frac{f(x)}{g(x)}\right)' = \frac{f'(x)g(x)- f(x)g'(x)}{g(x)^2}\]
\end{lema}

\begin{izrek}[Lagrange]
Naj bo $f$ zvezna ter odvedljiva na $[a,b]$. Potem obstaja $\lambda \in [a,b]$, da je \[f(b) - f(a) = f'(\lambda)(b-a)\]
\end{izrek}
\begin{oris}
$\varphi(x) = f(x) - f(a) + A(x-a)$ za tak $A$, da je $\varphi(b)=0$. Funkcija je zvezna na kompaktu, posledično ima minimum ter maksimum.
\end{oris}

\begin{izrek}[Odvod ima lastnost vmesne vrednosti]
Naj bo $f:[a,b] \to \R$ odvedljiva ter $f'(a) < f'(b)$. Potem za vsak $c \in [f'(a),f'(b)]$ obstaja $\lambda$, da je $f'(\lambda) = c$.
\end{izrek}
\begin{oris}
Za $c \in \{f'(a),f'(b)\}$ očitno, za vmesne $c$ vpeljemo zvezno funkcijo $\varphi(x) = f(x) - cx$, ki na kompaktu doseže minimum ter maksimum. 
\end{oris}

\subsection{Geometrija odvoda}
\begin{definicija}
\begin{itemize}
    \item $f$ ima v $c$ lokalni ekstrem, če v neki $\varepsilon$ okolici $c$ ni nobene točke, ki ima sliko večjo od $f(c)$. 
    \item Odvedljiva funkcija ima v lokalnem ekstremu ničlo odvoda.
    \item Predznak prvega odvoda določi padajočost/naraščajočost.
\end{itemize}
\end{definicija}

\newpage
\section{Integral}
\subsection{Nedoločeni integral}
Zaradi estetskih razlogov se bomo izognili pisanju $+C$ po vsakem nedoločenem integralu.
\begin{naloga}[Standardni]
\[
\int \frac{dx}{x} = \ln(x) \quad \quad \int \frac{dx}{\sin(x)^2} = -\text{ctg}(x) \quad \quad \int \frac{dx}{\cos(x)^2} = \tan(x) \\
\]
\end{naloga}

\begin{naloga}[Koreni]
\[\int \frac{dx}{\sqrt{a^2-x^2}} = \arcsin{\left(\frac{x}{a}\right)}\]
Pri naslednjih dveh integralih privzamemo $a>0$:
\begin{gather*}
\int \frac{dx}{\sqrt{a^2+x^2}} = \ln{\abs{x + \sqrt{x^2+a^2}}} \quad \quad \quad \int \frac{dx}{\sqrt{x^2-a^2}} = \ln{\abs{x+ \sqrt{x^2-a^2}}} \\
\end{gather*}
Sedaj ne zahtevamo več $a>0$.
\begin{gather*}
\int \sqrt{a^2 + x^2} \: dx = \frac{1}{2} \cdot \left(x \sqrt{a^2 + x^2} + a^2 \log(\sqrt{a^2 + x^2} + x)\right)\\
\int \sqrt{a^2 - x^2} \: dx = \frac{1}{2} \cdot \left( x \sqrt{a^2 - x^2} + a^2 \arctan(\frac{x}{\sqrt{a^2 - x^2}}\right)\\
\end{gather*}
\end{naloga}
\begin{naloga}[Obratne vrednosti kvadratov]
\[
\int \frac{dx}{a^2x^2+b^2} = \frac{1}{ab}\arctan\left(\frac{ax}{b}\right)
\]

\[
\int \frac{dx}{b^2-a^2x^2} = \frac{1}{2ab} \cdot \left(\log\left(\frac{ax}{b} + 1\right) - \log\left(1 - \frac{ax}{b}\right)\right)
\]
\end{naloga}

\begin{naloga}[Integriranje racionalnih funkcij]
Imenovalec racionalne funkcije razcepimo na produkt potenc linearnih in kvadratnih členov, potem pričnemo razcep na parcialne ulomke na naslednji način: \newline Vsak faktor $\frac{1}{(x-a)^{k}}$ v imenovalcu za ustrezne konstante $A_1, \dots, A_k$ prispeva:
\[\sum_{i=1}^{k} \frac{A_i}{(x-a)^i}\]
Vsak faktor $\frac{1}{(x^2+bx+c)^{l}}$ v imenovalcu za konstante $B_1, \dots, B_l$ ter $C_1, \dots, C_l$ prispeva:
\[\sum_{i=1}^{l} \frac{B_i x + C_i}{(x^2 + bx +c)^i}\]

Za integriranje parcialnih ulomkov uporabimo naslednje enakosti:
\[
\int \frac{A}{x-a} \: dx = A \ln{\abs{x-a}} \: \: \text{ter} \: \: \int \frac{A}{(x-a)^n} \: dx = \frac{-A}{n-1} \frac{1}{(x-a)^{n-1}} \: \: \text{za} \: n > 1
\]

\[
\int \frac{Bx+C}{x^2+bx+c} \: dx = \frac{B}{2} \ln{\abs{x^2+bx+c}} + \frac{2C-Bb}{\sqrt{-D}}\arctan\left(\frac{2x+b}{\sqrt{-D}}\right)
\]

\[
\int \frac{Bx+C}{(x^2+bx+c)^2} \: dx = \frac{(2C-Bb)x+(bC-2Bc)}{(-D)(x^2+bx+c)} + \frac{2(2C-Bb)}{(-D)\sqrt{-D}}\arctan\left(\frac{2x+b}{\sqrt{-D}}\right)
\]
\newline
kjer je $D = b^2-4c < 0$, saj privzamemo, da so kvadratni faktorji nerazcepni nad $\R$.
\end{naloga}

\begin{naloga}[Trigonometrične funkcije]
V splošnem lahko vsak integral vsebujoč racionalne kombinacije trigonometričnih funkcij prevedemo na integral racionalnih funkcij upoštevajoč naslednje identitete:
\[
t = \tan\left(\frac{x}{2}\right) \implies
\sin(x) = \frac{2t}{1+t^2}, \quad \cos(x) = \frac{1-t^2}{1+t^2}, \: \quad dx = \frac{2t}{1+t^2} dt
\]
Če funkciji $\sin$ ter $\cos$ nastopata v integralu kot potenci, večji od $1$, lahko integral poenostavimo z uporabo enakosti dvojnih kotov.

\bigskip

Pri integralih oblike $\int \sin(x)^p \cos(x)^q \: dx$ uporabimo naslednji recept:
\begin{itemize}
    \item $q$ je lih $\implies$ substitucija $t = \cos(x)$
    \item $p$ je lih $\implies$ substitucija $t = \sin(x)$
    \item $p$ in $q$ sta soda $\implies$ nižanje stopnje z uporabo formul o dvojnih kotih
\end{itemize}
\end{naloga}

%\begin{naloga}[Sporadični trigonometrični integrali]
%\[
%\int \frac{dx}{\sin(x)} = \int \frac{1}{t(1-t^2)} \: dt \quad \text{za} \: \: t = \sin\left(\frac{x}{2}\right)
%\]
%\end{naloga}

\begin{naloga}[Posebne vrste racionalnih funkcij]
\begin{itemize}
    \item Integral oblike $$\int R(x, \left( \frac{ax+b}{cx+d} \right)^{\frac{m}{n}}) \: dx \: \: \text{prevedemo na} \: \: \int R(\phi(t), t^m) \: dt \:,$$ kjer je $t = \left( \frac{ax+b}{cx+d} \right)^{\frac{1}{n}} = \phi^{-1}(x)$ ter je $\phi(t) = x = \frac{-dt^n+b}{ct^n-a}$
    \bigskip
    \item Za sledeči integral uporabimo nastavek: $$\int \frac{p(x)}{\sqrt{ax^2+bx+c}} \: dx = \tilde{p}(x) + \int \frac{C}{\sqrt{ax^2+bx+c}}\: dt \:,$$ kjer je $C$ konstanta, $\tilde{p}$ pa polinom stopnje ena manj kot $p$
    \bigskip
    \item Integrale oblike $$\int R(x, \sqrt{ax^2+bx+c}) \: dx$$ poenostavimo z uvedbo nove spremenljivke $u$:
    \begin{itemize}
        \item[$\dagger$] Če je $a > 0$: $\sqrt{a}(u-x) = \sqrt{ax^2+bx+c}$
        \item[$\dagger$] Če je $a < 0$ ter $x_1$ ničla kvadratne funkcije: $\sqrt{-a}(x-x_1)u = \sqrt{ax^2+bx+c}$
    \end{itemize}
    \item Integrale oblike: $$\int \frac{dx}{(x+\alpha)^k \sqrt{ax^2+bx+c}}$$ poenostavimo z uvedbo nove spremenljivke $t = \frac{1}{x+\alpha}$
\end{itemize}
\end{naloga}

\subsection{Določeni integral}

\begin{definicija}
$f:[a,b] \to \R$ je \textit{Riemmanovo integrabilna}, če obstaja $I$, da: $\forall \epsilon > 0$.$\exists \delta > 0$, da za vse delitve intervala $[a,b]$, drobnejše od $\delta$, ter vse usklajene izbire testnih točk velja: $$\abs{\sum_{i=1}^{n} f(t_i)\delta_i - I} < \epsilon$$

Omejena funkcija $f:[a,b] \to \R$ je \textit{Darbouxovo integrabilna}, če sta supremum spodnjih Darbouxovih vsot ter infimum zgornjih Darbouxovih vsot enaka:
\begin{align*}
\sup\{s(D) = \sum_{i=1}^{n} \inf\{f(x) \: | \: x \in [x_{i-1}, x_i]\} \cdot  \delta_i \} = \\
= \inf\{S(D) = \sum_{i=1}^{n} \sup\{f(x) \: | \: x \in [x_{i-1}, x_i]\} \cdot  \delta_i \},
\end{align*}
kjer supremum ter infimum vzamemo po vseh delitvah intervala $[a,b]$, ter je delitev $$D = [a = x_0, x_1, \dots, x_n = b]$$
\end{definicija}

\begin{izrek}[Lastnosti Darbouxovo ter Riemmanovo integrabilnih funkcij]
\begin{itemize}
%    \item Funkcija je Darbouxovo integrabilna natanko tedaj ko je Riemmanovo integrabilna.
    \item $f:[a,b] \to \R$ integrabilna $\implies$ $f$ na $[a,b]$ omejena.
    \item Zvezne funkcije na kompaktu ter monotone funkcije na kompaktu so integrabilne.
    \item $g$ zvezna ter $f$ integrabilna, obe na kompaktu $\implies$ $g \circ f$ je integrabilna.
    \item $$f(x) \leq g(x) \implies \int_{a}^{b} f(x) \: dx \leq \int_{a}^{b} g(x) \: dx$$
$$\abs{\int_{a}^{b} f(x) \: dx} \leq \int_{a}^{b} \abs{f(x)} \: dx$$
\end{itemize}
\end{izrek}

\begin{izrek}[Osnovna izreka]
$f$ integrabilna na $[a,b]$. Potem je funkcija $$F(x) = \int_{a}^{x} f(t) \: dt$$ zvezna na $[a,b]$ ter vsaka točka zveznosti $f$ da točko odvedljivosti $F$.

Če ima $f$ primitivno funkcijo povsod na $[a,b]$ velja: $$\int_{a}^{b} f(x) \: dx = F(b) - F(a)$$
\end{izrek}

\subsection{Posplošeni integral}
\begin{definicija}
$f:(a,b] \to \R$ integrabilna na $(t,b]$ za vsak $t \in (a,b)$. Potem je $$\int_{a}^{b} f(x) \: dx= \lim_{t \downarrow a} \int_{t}^{b}f(x) \: dx$$
\end{definicija}

\begin{izrek}[Integralski kriterij]
Naj bo $f:[1, \infty] \to \R$ \textbf{monotono padajoča ter nenegativna}. Potem:
$$\int_{1}^{\infty} f(x) \: dx \: \text{konvergira} \iff \sum_{i=1}^{\infty} f(i) \: dx \: \text{konvergira}$$
\end{izrek}
\begin{oris}
Trivialen sendvič z Riemmanovo vsoto.
\end{oris}

\begin{izrek}
Naj bo $f:[a,b) \to \R$ integrabilna na vsakem intervalu $[a,t)$. Potem $$\int_{a}^{b}\abs{f(x)} \: dx \; \; \text{obstaja} \implies \int_{a}^{b}f(x) \: dx \; \; \text{obstaja}$$
\end{izrek}
\begin{oris}
$F(x) = \int_{a}^{x} f(x) \: dx$ ter $G(x) = \int_{a}^{x} g(x) \: dx$ sta zvezni na $[a,b)$ ter obstaja $\lim_{x \uparrow b}G(x)$. Uporabimo Cauchyjev pogoj za obstoj limite ter trikotniško neenakost za integrale.
\end{oris}

\begin{izrek}
Naj bo f integrabilna na $[a,b].$
\begin{itemize}
    \item $$s < 1 \implies \int_{a}^{b} \frac{f(x)}{(x-a)^s} \: dx \; \text{obstaja}$$
    \item Če $s \geq 1$ ter obstja $m \in \R^{+}$, da je izpolnjen en izmed pogojev:
        \begin{itemize}
            \item[$\dagger$] $\forall x \in [a,b]: f(x) \geq m$
            \item[$\dagger$] $\forall x \in [a,b]: f(x) \leq -m$
        \end{itemize}
        $$\implies \int_{a}^{b} \frac{f(x)}{(x-a)^s} \: dx = \infty$$
\end{itemize}
\end{izrek} 
\begin{oris}
Zamenjamo $\abs{f(x)}$ z $m$ ter uporabimo integralski kriterij.
\end{oris}

\begin{izrek}[Integrabilnost v $\pm \infty$]
Naj bo $f$ integrabilna na vsakem intervalu $[a,b]$, $b > a$. Potem:$$\int_{a}^{\infty} f(x) \: dx \;\text{obstaja} \iff \forall \varepsilon > 0. \; \exists B \in \R: \: \forall \; b,b' > B: \abs{\int_{b'}^{b} f(x) \: dx} < \varepsilon$$    
\end{izrek}
\begin{oris}
Preidemo na primitivno funkcijo ter uporabimo Cauchyjev pogoj za limite v neskončnosti.
\end{oris}

\begin{izrek}
Naj bo $g: [a, \infty] \to \R$ zvezna za $a > 0$.
\begin{itemize}
    \item $$g \: \text{omejena} \: D_g \land p > 1 \implies \int_{a}^{infty} \frac{g(x)}{x^p} \: dx \: \text{konvergira}$$
    \item Če $p \leq 1$ ter obstaja $m \in \R^{+}$, da je izpolnjen eden izmed pogojev:
    \begin{itemize}
            \item[$\dagger$] $\forall x \in [a,b]: g(x) \geq m$
            \item[$\dagger$] $\forall x \in [a,b]: g(x) \leq -m$
    \end{itemize}
    $$\implies \int_{a}^{\infty} \frac{g(x)}{x^p} \: dx = \infty$$
\end{itemize}
\end{izrek}

\subsection{Geometrija integrala}
Ne da se mi.

\newpage
\section{Funkcijska zaporedja in vrste}
\begin{definicija}
Podobno podpoglavjema o zapordjih in vrstah realnih števil, le da uporabljamo metriko: $$d_{\infty}(f,g) = \sup_{x \in I} \abs{f(x) - g(x)}$$
Konvergenci glede na to metriko pravimo enakomerna konvergenca.
\end{definicija}

\begin{izrek}[Lastnosti enakomerne konvergence]
\begin{itemize}
    \item \textbf{Metrični prostor funkcij z metriko $d_\infty$ je poln.}
    \item Če zvezne funkcije $\{f_n\}$ konvergirajo enakomerno k $f$ $\implies$ $f$ zvezna.
\end{itemize}    
\end{izrek}

\begin{izrek}[Weierstraßov kriterij]
$\{f_n\}$ funkcijsko zaporedje ter obstaja zaporedje pozitivnih števil $\{m_n\}$, da $f_n(x) \leq m_n,$ za vse $n$ ter vse $x \in I$. $$\sum_{i=1}^{\infty}m_i \; \text{konvergira} \implies \sum_{i=1}^{\infty}f_n(x) \; \text{konvergira enakomerno}$$
\end{izrek}
\begin{oris}
Konvergenca po točkah očitna, uporabimo Cauchyjev kriterij za enakomerno konvergenco.
\end{oris}

\subsection{Integriranje in odvajanje funkcijskih zaporedij}
\begin{izrek}[Lastnosti enakomerne konvergence]
\begin{itemize}
    \item Če zvezne funkcije $\{f_n\} \in C^0$ konvergirajo enakomerno proti $f$, potem je $$\lim_{n \to \infty} \int_{a}^{b}f_n(x) \: dx = \int_a^b f(x) \: dx .$$
    \item Zvezno odvedljive $\{f_n: [a,b] \to \R\}$ ter $\{f_n(c)\}$ konvergira za nek $c \in [a,b]$. \newline Če $\{f_n' \xrightarrow[]{} g\}$ enakomerno, potem $\{f_n \xrightarrow[]{} f\}$ enakomerno ter velja $$f'(x) = \lim_{n \to \infty}f_n(x)'$$
\end{itemize}    
\end{izrek}
Ker zgornji izreki veljajo za enakomerno konvergentna zaporedja, veljajo tudi za enakomerno konvergentne vrste.

\newpage
\subsection{Potenčne vrste}

\begin{izrek}[Obstoj konvergenčnega radija][obstoj konv. radija]
Za potenčno vrsto $\vst{\infty} a_n(x-c)^n$ obstaja $R \in [0, \infty]$ z lastnostjo: \begin{itemize}
    \item za $x \in (c-R,c+R)$ je vrsta konvergentna, ter divergentna za  $\abs{x-c} > R$, v točkah $\abs{x-c} = R$ pa bodisi konvergira, bodisi divergira
    \item Vrsta konvergira absolutno in enakomerno na $[c-r, c+r]$ za vsak $r < R$.
\end{itemize}
\end{izrek}
\begin{oris}
Prva točka sledi po supremumu, druga pa upoštevajoč, da če vrsta konvergira pri $x = x_0$, potem $\lim_{n \to \infty} \abs{a_n x^n} = 0 \implies \abs{a_n x^n < M}$ ter $\abs{a_n x^n} < \abs{a_n \left(\frac{r}{x_0}\right)^n} \leq M \left( \frac{r}{x_0} \right)^n$, zaključimo z Weierstraßom.
\end{oris}
\begin{izrek}[Cauchy-Hadamard]
Naj bo $\vst{\infty} a_n (x-c)^n$ potenčna vrsta. Za konvergenčni polmer $R$ velja:
\begin{itemize}
    \item $\frac{1}{R} = \lim_{n \to \infty} \frac{\abs{a_{n+1}}}{a_n}$, če ta limita obstaja.
    \item $\frac{1}{R} = \lim_{n \to \infty} \sqrt[n]{a_n}$, če ta limita obstaja.
    \item $$\frac{1}{R} = \limsup_{n \to \infty} \sqrt[n]{a_n}$$
\end{itemize}
\end{izrek}
\begin{oris}
Za prva dva očitno. Če je $\limsup_{n \to \infty} \sqrt[n]{a_n} = \infty$ hitro dobiš neničelnost limite členov. Če je $\limsup_{n \to \infty} \sqrt[n]{a_n} = A$, potem najdemo $q$, da je $\abs{x} < \frac{1}{q} < \frac{1}{a}$, po definiciji $\limsup$ za vse $n > N$ velja: $\sqrt[n]{a_n} < q$, ker je $\abs{qx} < 1$ smo pokazali da vrsta konvergira. Če je $A > \frac{1}{\abs{x}}$ obstaja neskončno $n$, da je $\sqrt[n]{a_n} > \frac{1}{\abs{x}} \implies \abs{a_n x^n} > 1$.
\end{oris}

\begin{izrek}[Zveznost, odvedljivost ter integrabilnost]
Po izreku~\ref{izrek:obstoj konv. radija} vemo, da je vsosta potenčne vrste s konvergenčnim radijem $R > 0$ \textbf{zvezna} funkcija na $(c-R, c+R)$. O krajiščih pa govori \textit{Abelov izrek:}
\[
\text{Če } \vst{\infty} a_n(x)^n \text{ konvergira pri }x = \pm R,\text{ potem je vsota zvezna pri }x = \pm R.
\]
Naj bo $R > 0$ konvergenčni polmer $f(x) = \vst{\infty}a_i x^i$. Potem imata vrsti, ki jih dobimo z členoma odvajanjem in integriranjem $f$ prav tako konvergenčni polmer $R$ ter za vse $x \in (-R,R)$ velja: $$f'(x) = \vst{\infty} n a_n x^{n-1} \quad \text{in} \quad \int_{0}^{x}f(t) \: dt = \vst{\infty} \frac{a_n}{n+1}x^{n+1}$$
\end{izrek}
\begin{oris}
Drugi del trivialen, $R$ ostaja konvergenčni radij, ker je $\lim_{n \to \infty} \sqrt[n]{n} = 1$
\end{oris}

\subsection{Taylorjeva vrsta}
\begin{lema}
Če je $P$ polinom stopnje $n$ velja:
\[
P(x+y) = P(x) + \frac{P'(x)}{1!}y + \frac{P''(x)}{2!}y^2 + \dots + \frac{P^{\left(n \right)}(x)}{n!}y^n = \sum_{i=0}^{n} \frac{P^{\left( i \right)}(x)}{i!}y^i
\]
\end{lema}
\begin{definicija}[Taylorjev polinom]
$f$ je $n$-krat odvedljiva v okolici $a$. \textbf{$n$-ti Taylorjev polinom funkcije $f$ pri $a$} je
\[
T_{n,a}(x) = \sum_{i=0}^n \frac{f^{\left( i \right)}(a)}{i!}(x-a)^i
\]
Če je $f$ neskončnokrat odvedljiva v okolici $a$ lahko $f$ priredimo \textbf{Taylorjevo vrsto}:
\[
\sum_{i=0}^\infty \frac{f^{\left( i \right)}(a)}{i!}(x-a)^i = \lim_{n \to \infty} T_{n,a}(x),
\]
če slednja obstaja.
\end{definicija}
Hitra posledica je, da če je $f$ vsota konvergentne potenčne vrste, potem je $f$ vsota prirejene Taylorjeve vrste.

Zanima nas obnašanje $R_n(x) = f(x) - T_n(x)$.
\begin{izrek}[Taylorjev izrek]
Naj bo $f$ odvedljiva $(n+1)$-krat na odprtem intervalu $I$, ki vsebuje $a$. Za vsaj $x \in I$ obstaja $c$ med $a$ in $x$, da velja:
\[
R_{n,a}(x) = \frac{f^{\left( n+1 \right)}(c)}{(n+1)!}(x-a)^{n+1}
\]
\end{izrek}
\begin{oris}
Za $0 \leq k \leq n$ velja $R_{n,a}^{(k)}(a) = 0$. Fiksiramo $x$ ter izberemo $s \in \R$, da je $R_{n,a}(x) = s(x-a)^{n+1}$. Uvedemo $G(y) = R_{n,a}(y) - s(y-a)^{n+1}$, velja $G(x) = 0$ ter $\forall 0 \leq k \leq n: G^{(k)}(a) = 0$. Rezultat sledi po $n$-kratni uporabi Rollovega izreka, upoštevajoč definiciji $G$ ter $R_{n,a}$.
\end{oris}
Moč Taylorjevega izreka leži v tem, da lahko pogosto omejimo vrednosti, ki jih koeficient $\frac{f^{\left( n+1 \right)}(c)}{(n+1)!}$ zavzame med $a$ in $x$, kar nam omogoča vspostaviti uporabne neenakosti.
\begin{naloga}
Opazujmo
$$
f(x)=
\begin{cases}
e^{\frac{-1}{x}}; \: x > 0\\
0; \: x \leq 0
\end{cases}
\quad f'(x)=
\begin{cases}
\frac{1}{x^2}e^{\frac{-1}{x}}; \: x > 0\\
0; \: x \leq 0
\end{cases}
$$
Ker je $\lim_{\delta \to 0} \delta^{-1} e^{\frac{1}{\delta}} = 0 $ je očitno $f \in C^{\infty}$ ter $f^{(n)}(0)= 0 \: \forall n$. Prirejena Taylorjeva vrsta $f$ v $0$ je potlej enaka ničelni in konvergira povsod, njena vsota pa ni $f$. Zato uvedemo naslednji razred funkcij.
\end{naloga}

\begin{definicija}[Analitične funkcije]
$f: I \to \R$ je \textbf{realno analitična} na intervalu $I$ (označimo $f \in C^{\omega}(I)$), če $\forall a \in I. \: \exists r_a > 0. \: (a-r_a, a+r_a) \subseteq I$ in je $f$ na $(a-r_a,a+r_a)$ vsota konvergentne potenčne vrste:
\[
f(x) = \sum_{k=0}^{\infty} c_k (x-a)^k , \: x \in (a-r_a,a+r_a)
\]
\end{definicija}

\begin{izrek}[Splošni Taylorjev izrek]
Naj bo $I$ odprt interval vsebujoč $a$ in $f \in \C^{(n+1)}(I)$. Za vsak $x \in I$ in $p \in \N$ obstaja $c$ med $x$ in $a$, da za $f(x) = T_{n,a}(x) + R_{n,a}(x)$ velja:
\[
R_{n,a}(x) = \frac{f^{(n+1)(c)}}{p \cdot n!}(x-a)^p(x-c)^{n-p+1}
\]
V posebnih primerih velja:
\begin{itemize}
    \item $p=n+1$: $$ R_{n,a}(x) = \frac{f^{(n+1)(c)}}{(n+1)!}(x-a)^{n+1}$$
    \item $p=1$: $$R_{n,a}(x) = \frac{f^{(n+1)(c)}}{n!} (x-a)(x-c)^n$$
\end{itemize}
\end{izrek}
\begin{oris}
Fiksiramo $x \in I$ ter izberemo poljubne $b \in I$ ter $p \in \N$. Definiramo $\varphi(x) = T_{n,x}(b) + \left(\frac{b-x}{b-a}\right)^p R_{n,a}(b)$, velja $\varphi(x) \in C^1(I)$. Ker je $f(a) = f(b)$ rezultat sledi po Rollejevem izreku, saj je:
\begin{multline*}
\varphi'(x) = f'(x) - f'(x) + (b-x)f''(x) - (b-x)f''(x) + \dots + \frac{(b-x)^n}{n!}f^{(n+1)}(x) \\
+ (-p) \left(\frac{(b-x)^{p-1}}{(b-a)^{p}}\right) R_{n,a}(b)\\
\implies R_{n,a}(b) = \frac{(b-x)^{n-p+1} (b-a)^p}{p \cdot n!} f^{(n+1)}(c)
\end{multline*}
\end{oris}

\begin{naloga}[Taylorjeve vrste osnovnih funkcij]
\begin{itemize}
    \item $f(x) = e^x$ s konvergenčnim radijem $\infty$: $$e^x = \sum_{k=0}^{\infty} \frac{1}{k!}x^k \quad \quad \forall x \in \R$$
    \item $f(x) = \sin(x)$ s konvergenčnim radijem $\infty$: $$\sin(x)=  \sum_{k=0}^\infty \frac{(-1)^k}{(2k+1)!}x^{2k+1} \quad \quad \forall x \in \R$$
    \item $f(x) = \cos(x)$ s konvergenčnim radijem $\infty$: $$\cos(x)=  \sum_{k=0}^\infty \frac{(-1)^k}{(2k)!}x^{2k} \quad \quad \forall x \in \R$$
    \item $f(x) = \ln(1+x)$ s konvergenčnim radijem $1$: $$\ln(1+x) = \sum_{k=0}^{\infty} \frac{(-1)^k}{k+1} x^{k+1} \quad \quad \forall x \in (-1,1)$$
    \item $f(x) = (1+x)^\alpha$ za $\alpha \in \R$, kjer je $\binom{\alpha}{k} = \frac{\alpha(\alpha-1)\dots(\alpha-k+1)}{k!}$: $$(1+x)^\alpha = \sum_{k=0}^\infty \binom{\alpha}{k}x^k \quad \quad x \in (-1,1)$$
    \item logaritem v okolici $a \in \R^+$, ki konvergira za $\abs{\frac{x-a}{a}} < 1 \implies 0 < x < 2a$: 
    \begin{multline*}
    \log(x) = \log(a+x-a) = \log(a) + \log\left(1+\frac{x-a}{a}\right)=\\ = \log(a) + \sum_{k=0}^\infty \frac{(-1)^k}{k+1}\left( \frac{x-a}{a}\right)^k \quad \quad x \in (0,2a) \quad \quad
    \end{multline*}
\end{itemize}
\end{naloga}


\newpage
\section{Metrični prostori}
%\newpage
%\printbibliography
%\addcontentsline{toc}{section}{Literatura}

\end{document}
